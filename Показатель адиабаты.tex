\documentclass[a4paper,14pt]{article}
\usepackage[utf8]{inputenc}	
\usepackage[10pt]{extsizes}
\usepackage{indentfirst}
\title{Лабораторная работа 2.1.2\\Определение $\frac{c_p}{c_v}$ методом
изобарного расширения газа}
\author{Выполнил: Александров Никита\\ Студент первого курса ФРКТ\\Группа Б01-107}
\vspace{20cm}
\date{МФТИ 2022 г.}
\usepackage[utf8]{inputenc}
\usepackage[T2A]{fontenc}
\usepackage[english, russian]{babel}
\usepackage{amsthm}
\usepackage{floatrow}
\usepackage{amsmath}
\usepackage{amssymb}
\usepackage{caption}
\usepackage{tikz}
\usepackage{ dsfont }
\usepackage{wrapfig}
\usepackage{graphicx}
\usepackage{textcomp}
\usepackage{marvosym}
\usepackage{ esint }
\usepackage{setspace}
\onehalfspacing
\setlength{\topmargin}{-0.5in}
\setlength{\textheight}{9.1in}
\setlength{\oddsidemargin}{-0.4in}
\setlength{\evensidemargin}{-0.4in}
\setlength{\textwidth}{7in}
\setlength{\parindent}{5ex}
\setlength{\parskip}{0cm}


\begin{document}
\selectlanguage{russian}
\maketitle
\newpage
\begin{center}
\tableofcontents
\end{center}
\newpage

\begin{center}
\section*{Аннотация}
\end{center}

В данной работе вычислен показатель адиабаты для воздуха $\gamma = \frac{C_P}{C_V}$ методом изобарического расширения при следующих условиях: $T \approx 23^{\circ}C$, $p \approx p_\text{атм}$ $V_\text{возд} \approx 20$ л. Сделан вывод о применимости данного метода к вычислению показателя адиабаты воздуха в данных условиях. play with dick


\section{Введение}

В данной работе показатель адиабаты вычислен методом изобарического расширения. Этот метод имеет ряд приемуществ по сравнению с методом адиабатического расширения. В теоретическом описании данного метода использовано уравнение теплопроводности, что позволяет количиственно учтесть влияние времени расширения на итоговый результат, благодаря этому можно дать точную количественную оценку для $\gamma$; в то время как в методе адиабатического расширения время процесса в явном виде не входит в формулы, из-за чего его влияние нельзя оценить напрямую, точной формулы для $\gamma$ нет, поэтомк экспериментатор ограничен лишь косвенной оценкой этой величины. Именно по этой причине для измерения выичсления показателя адиабаты выбран метод изобарического расширения.

\section{Экспериментальная установка и теоретические сведения}

Экспериментальная установка включает себя: сосуд с воздухом, два крана: кран К1 для создания избыточного давления воздуха и кран К2, соединяющий сосуд с атмосферой, водный манометр. Основной эксперимент заключается в следующем: с помощью насоса, соединённого с сосудом с помощью К1 создаётся избыточное давление воздуха в сосуде, затем кран перекрывается, после этого происходит процесс изохорного остывания, конечная разность давления $\Delta h_1 = h_{1\text{l}} - h_{1\text{r}}$ (l - left, r - right - левый и правый столбы манометра соответственно) фиксируется. Затем на время $t$ открывается кран К2 и происходит сначала процесс адиабатического сжатия, время которого мало по сравнению со временем следующего за ним процесса изобарного расширения. Конечное избыточное давление 
$\Delta h_2 = h_{2\text{l}} - h_{2\text{r}}$ также фиксируется.

Запишем уравнения, описывающие наш процесс: в данном случае воздух можно считать идеальным газом, поэтому для него верно уравнение Менделеева-Клайперона \begin{equation}
m = \frac{P_0 V_0}{RT} \cdot \mu
\end{equation}

Также все описанные процессы являются политропическими, поэтому верно уравнение политропы для адиабатического процесса: \begin{equation}
\frac{T^{\gamma}}{P^{\gamma - 1}} = const
\end{equation}

Помимо этого верно уравнение теплового баланса (теплопроводности): \begin{equation}
c_p m dT = -\alpha(T-T_0)dt
\end{equation}

Решая дифференциальное уравнение (3), используя уравнения (1) и (2), проводя необходимые преобразования и подстановки, имеем уравнение: \begin{equation}
ln\left(\frac{\Delta h_1}{\Delta h_2}\right) = ln\left(\frac{\gamma}{\gamma - 1}\right) + \left(\frac{\alpha}{c_p m_0}\right) \cdot t
\end{equation}

Таким образом, исследуя зависимость $\frac{\Delta h_1}{\Delta h_2}$ от времени $t$, можно получить значение показателя адиабаты: $ln\left(\frac{\gamma}{\gamma - 1}\right)$ является свободным коэффициентом графика $ln\left(\frac{\Delta h_1}{\Delta h_2}\right) (t)$.
\newpage

\section{Основная часть}
\subsection{Основной эксперимент}

Изначально зафиксированы параметры, при которых ведётся эксперимент: $h_l = h_r = 178$ мм, то есть исходное избыточное давление воздуха равно нулю; $T_0 = 23,2^{\circ}C$, в течение эксперимента эта температура понизилась до значения $T_{\text{кон}} = 23^{\circ}C$. После проверки исправности установки проведён основной эксперимент, все данные занесены в таблицу.

\begin{table}[h]
\caption{Результаты основного эксперимента}
\label{}
\begin{tabular}{|c|c|c|c|c|c|c|c|}
\hline
№ опыта & 1 & 2     & 3     & 4     & 5     & 6     & 7     \\ \hline
$h_{r\text{l}}$, мм & 103                      & 119   & 113   & 120   & 115   & 116   & 97    \\ \hline
$h_{1\text{l}}$, мм & 254                      & 236   & 241   & 235   & 240   & 238   & 261   \\ \hline
$h_{2\text{l}}$, мм & 158                      & 164   & 167   & 170   & 172   & 174   & 174   \\ \hline
$h_{2\text{l}}$, мм & 197                      & 191   & 188   & 185   & 183   & 181   & 182   \\ \hline
$\Delta h_1$, мм & 151                      & 117   & 128   & 115   & 125   & 122   & 164   \\ \hline
$\Delta h_2$, мм & 39                       & 27    & 21    & 15    & 11    & 7     & 8     \\ \hline
$t$, с    & 5,25                     & 10,07 & 14,91 & 20,01 & 25,17 & 30,12 & 35,05 \\ \hline
\end{tabular}
\end{table}

На основании этих данных построен график и вычислена погрешность итогового результата. Поскольку формула (4) имеет достаточно громоздкую структуру, то вычисление погрешностей является достаточно трудной и важной задачей, поэтому их расчёты вынесены в отдельный раздел.

\subsection{Вычисление погрешностей}

1. Погрешность времени в этой работе принимается равной одной секунде, довольно трудно точно измерить время, при котором кран был открыт, поэтому погрешность берётся довольно большой.

2. Рассмотрим погрешность величины $ln\left(\frac{\Delta h_1}{\Delta h_2}\right) \equiv \xi = ln(A)$, где $A = \frac{\Delta h_1}{\Delta h_2}$: её можно рассчитать с помощью многократного применения формулы для погрешности косвенной величины: $\sigma_{\xi} = \frac{d \xi}{dA} \cdot \sigma_{A} = \frac{d(lnA)}{dA} \cdot \sigma_{\xi} = \frac{\sigma_A}{A}$; заранее рассчитаем $\sigma(\Delta h_i) = \sqrt{(\sigma_{hil})^2 + ((\sigma_{hir}))^2}$ $=$ $\sqrt{2} \cdot \sigma_h$
 \begin{equation}
A = \frac{\Delta h_1}{\Delta h_2}, \sigma_A = A \cdot \sqrt{\left(\frac{\sigma(\Delta h_1)}{\Delta h_1}\right)^2 + \left(\frac{\sigma(\Delta h_2)}{\Delta h_2}\right)^2} = \sqrt{2} \cdot A \cdot \sigma_h \cdot \sqrt{\left(\frac{1}{\Delta h_1}\right)^2 + \left(\frac{1}{\Delta h_2}\right)^2}
\end{equation} 

Примем $\sigma_h = 1$мм, тогда подставляя полученное выражение в выражение для $\sigma_{\xi}$, получим:
\begin{equation}
\sigma_{\xi} = \sqrt{2} \cdot \sigma_h \cdot \sqrt{\left(\frac{1}{\Delta h_1}\right)^2 + \left(\frac{1}{\Delta h_2}\right)^2} = \sqrt{2} \cdot \sqrt{\left(\frac{1}{\Delta h_1}\right)^2 + \left(\frac{1}{\Delta h_2}\right)^2} \text{мм}
\end{equation}

3. Теперь проведём расчёт погрешности величины $\gamma$: пусть после аппроксимации было получено значение $B \pm \sigma_B$, тогда  \begin{equation}
ln\left(\frac{\gamma}{\gamma - 1}\right) = B \rightarrow \frac{\gamma}{\gamma - 1} = exp(B) 
\rightarrow \frac{\gamma - 1}{\gamma} = exp(-B) \rightarrow \end{equation} 
\begin{equation}
\rightarrow 1 - \frac{1}{\gamma} = exp(-B) \rightarrow \gamma = \frac{1}{1 - exp (-B)}
\end{equation}
\begin{equation}
\sigma_{\gamma} = \frac{d\gamma}{dB} \cdot \sigma_{B} = \frac{(exp(-B))'}{(1 - exp(-B))^2} \cdot \sigma_{B} = \frac{-exp(-B)}{(1-exp(-B))^2} \cdot \sigma_{B}
\end{equation}

Таким образом, все погрешности вычислены, получившиеся формулы можно использовать для расчёта показателя адиабаты.

\subsection{Построение графика}
После расчёта погрешностей составлена сводная таблица, в которую занесены все данные, необходимые для построения графика и вычисления показателя адиабаты: \newpage

\begin{table}[h!]
\caption{Сводная таблица для графика}
\label{   }
\begin{tabular}{|c|c|c|c|c|c|}
\hline
$t$, с  & $\sigma_t$, с & $\varepsilon_t, \%$  & $ln\left(\frac{\Delta h_1}{\Delta h_2}\right)$ & $\sigma_{\xi}$ & $\varepsilon_{\xi}, \% $  \\ \hline
5,25  & 1      & 19,0   & 1,35       & 0,04     & 2,8     \\ \hline
10,07 & 1       & 9,9    & 1,47       & 0,05     & 3,7     \\ \hline
14,91 & 1       & 6,7    & 1,81       & 0,07     & 3,8     \\ \hline
20,01 & 1       & 5,0    & 2,04       & 0,10     & 4,7     \\ \hline
25,17 & 1       & 4,0    & 2,43       & 0,13     & 5,3     \\ \hline
30,12 & 1       & 3,3    & 2,86       & 0,20     & 7,1     \\ \hline
35,05 & 1       & 2,9    & 3,02       & 0,18     & 5,9     \\ \hline
\end{tabular}
\end{table}

\begin{center}
\begin{figure}[h!]
\floatsetup[wrapfigure]{capposition=bottom}
\includegraphics[scale=0.7]{1}
\label{1}
\end{figure} 
\end{center}

Полученное значение свободного коэффициента графика получилось равным $1,03 \pm 0,06$, по формуле (8): $\gamma = 1,56$, по формуле (9): $\sigma_{\gamma} = 0,05$, конечное значение показателя адиабаты: $\gamma = 1,56 \pm 0,05$, $\varepsilon_{\gamma} = 3,2 \%$.

\section{Анализ полученных результатов и выводы}
В данной работе вычислен показатель адиабаты воздуха $\gamma$ методом изобарного расширения, получившееся значение $\gamma = 1,56 \pm 0,05$, табличное значение $\gamma_{\text{табл}} = 1,4$. Вычисленное экспериментальным путём значение не очень хорошо совпало с табличным, так как табличное значение $\gamma_{\text{табл}}$ не лежит в пределах погрешности $\gamma_{\text{выч}}$. Тем не менее расхождение между $\gamma_{\text{выч}}$ и $\gamma_{\text{табл}}$ не велико и составляет порядка $7 \%$.

Из всего можно сделать следующий вывод: метод изобарного расширения применим для вычисления показателя адиабаты, однако точность данного метода не велика.

\end{document}