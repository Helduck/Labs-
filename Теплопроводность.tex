\documentclass[a4paper,14pt]{article}
\usepackage[utf8]{inputenc}	
\usepackage[10pt]{extsizes}
\usepackage{indentfirst}
\title{Отчёт о лабораторнй работе 2.2.3\\Измерение теплопроводности воздуха при атмосферном давлении}
\author{Выполнил: Александров Никита\\ Студент первого курса ФРКТ\\Группа Б01-107}	
\date{МФТИ 2022 г.}
\usepackage[utf8]{inputenc}
\usepackage[T2A]{fontenc}
\usepackage[english, russian]{babel}
\usepackage{amsthm}
\usepackage{floatrow}
\usepackage{amsmath}
\usepackage{amssymb}
\usepackage{caption}
\usepackage{tikz}
\usepackage{ dsfont }
\usepackage{wrapfig}
\usepackage{graphicx}
\usepackage{textcomp}
\usepackage{marvosym}
\usepackage{ esint }
\usepackage{setspace}
\onehalfspacing
\setlength{\topmargin}{-0.5in}
\setlength{\textheight}{9.1in}
\setlength{\oddsidemargin}{-0.4in}
\setlength{\evensidemargin}{-0.4in}
\setlength{\textwidth}{7in}
\setlength{\parindent}{5ex}
\setlength{\parskip}{0cm}

\begin{document}
\selectlanguage{russian}
\maketitle
\newpage
\begin{center}
\tableofcontents
\end{center}
\newpage

\begin{center}
\section*{Аннотация}
\end{center}

В данной работе измерена зависимость теплопроводности воздуха от температуры при атмосферном давлении в диапазоне температур от $20^{\circ}C$ до $70^{\circ}C$. Построены нагрузочные кривые $R(Q)$ и вычислены значения сопротивления нити при соответствующих температурах. Вычислен коэффициент $\frac{dR}{dT}$ для нити в данных условиях и коэффициент сопротивления материала нити $\alpha = \frac{1}{R_{273}} \cdot \frac{dR}{dT}$. Построен график зависимости $ln(K)(kn(T))$ и вычислен коэффициент пропорциональности $\beta$: $k \propto T^{\beta}$. Полученные результаты сопоставлены с табличными данными, сделаны выводы о применимости данного метода к измерению теплопроводности воздуха при атмосферном давлении в данных условиях.

\section{Введение}

В данной работе коэффициент пропорциональности между теплопроводностью и температурой вычислен с помощью построения нагрузочных кривых для $R(Q)$ для нагреваемой нити. Поскольку при пропускании через нить ток она нагревеатся, то её температура всегда выше, чем температура окружающей среды (в данном случае воздуха). Теоретическая зависимость сопротивления нити от мощности, выделяемой на ней, линейная $R = a \cdot Q + b$, измерение этой зависимости позволяет найти коэффициент $b = R(0)$. Проводя этот эксперимент для разных температур можно установить зависимость коэффициента теплопроводности воздуха от температуры, что и проделано в этой работе. Таким образом, нить служит как источником тепла, так и датчиком температуры. Именно поэтому для вычисления теплопроводности воздуха выбрана именно эта экспериментальная установка.

\section{Теоретические сведения и экспериментальная установка}

Теплопроводность - это процесс передачи тепловой энергии от нагретых частей системы к холодным за счёт хаотического движения частиц среды. Перенос тепла описыватеся законом Фурье: 
\begin{equation}
\vec{q} = - k \cdot \nabla T
\end{equation}
где $q$ - плотность потока энергии, $k$[Вт/м$\cdot$K] - коэффициент теплопроводности, $\nabla T$ - градиент температур. В экспериментальной установке используется длинный цилиндр, поэтому для него можно пренебречь теплоотводом через торцы, в данном случае $\nabla T = \frac{dT}{dr}$. Количество потока тепла через цилиндрическую поверхность $Q$ вычисляется по формуле $Q = -q \cdot S = k \cdot 2 \pi r L \cdot \frac{dT}{dr}$, где $S = 2 \pi r L$ - площадь поверхности циллиндра. Разделяя переменные в этом уравнении и интегрируя по радиусу, получим уравнение 

\begin{equation}
Q = \frac{2 \pi L}{ln\frac{r_0}{r_1}} \cdot k \cdot \Delta T \Rightarrow k = \frac{Q}{\Delta T} \cdot \frac{ln\frac{r_0}{r_1}}{2 \pi L}
\end{equation}

Где $r_0$ - радиус циллиндра, $r_1$ - радиус нити. Экспериментальная установка состоит из молибденовой нити диаметром $2r_1 \approx 0,05$ мм, длиной $L \approx 40$ см, сопротивление нити является однозначной функцией температуры: $R(t) = R_{273} \cdot (1 + \alpha t)$, где $t$ - температура в $^{\circ}C$, $\alpha$ - температурный коэффициент сопротивления материала нити. При $Q = 0$ сопротивление нити равно сопротивлению нити при температуре, равной температуре термостата. При обработке данных вычисляется зависимость $R(Q)$, а по ней вычисляется свободный коэффициент графика $R(0) = R(T)$ для каждой температуры. По этим данным вычисляются коэффициенты $\frac{dR}{dT}$ и $\alpha$. Для каждой температуры с помощью (1) вычисляется коэффициент теплопроводности $k$ по графику зависимости $ln(K)(kn(T))$ вычисляется $\beta$: $k \propto T^{\beta}$.

Мощность и сопротивление, выделяемые на нити вычисляются как $Q_{\text{нити}} = U_{\text{нити}} \cdot I_{\text{нити}}$, $R_{\text{нити}} = \frac{U_{\text{нити}}}{I_{\text{нити}}}$. напряжение нити измеряется вольтметром, а сила тока вычисляется с помощью эталонного сопротивления, которое известно заранее: мультиметром измеряется напряжение на эталонном сопротивлении, а поскольку нить и эталонный резистор соединены последовательно, то $I_{\text{нити}} = I_{\text{эт}} = \frac{U_{\text{эт}}}{R_{\text{эт}}}$.

Погрешность коэффициента теплопроводности $k$ можно вычислить как погрешность косвенного измерения по формуле \begin{Large} \begin{equation}
\sigma_k = k \cdot \sqrt{\left(\varepsilon_L\right)^2 + \left(\varepsilon_{ln\frac{r_0}{r_1}}\right)^2 + \left(\varepsilon_{\frac{dR}{dQ}}\right)^2 + \left(\varepsilon_{\frac{dR}{dT}}\right)^2}
\end{equation}
\end{Large}
Где \begin{Large} $\varepsilon_{ln\frac{r_0}{r_1}} = \frac{\sqrt{\left(\frac{\sigma_{r_1}}{r_1}\right)^2 + \left(\frac{\sigma_{r_0}}{r_0}\right)^2}}{ln\frac{r_0}{r_1}}$, $\varepsilon_L = \frac{\sigma_L}{L}$, $\varepsilon_{\frac{dR}{dT}} = \frac{\sigma_{\frac{dR}{dT}}}{\frac{dR}{dT}}$, $\varepsilon_{\frac{dR}{dT}} = \frac{\sigma_{\frac{dR}{dQ}}}{\frac{dR}{dQ}}$ \end{Large}. 

\section{Ход работы}
\subsection{Основной эксперимент}
В этой части работы измерена зависимость напряжения на нити и эталонном резистре от сопротивления $R_{\text{м}}$ - магазинного сопротивления. Точность измерений составляет 4-5 знаков после запятой, мультиметр, используемый в этой работе, способен измерять напряжение с такой точностью. Результаты измерений и вычислений занесены в таблицы. В таблицах используются следующие обозначения: $R_{\text{м}}$ - сопротивление реостата (в данном случае магазина сопротивлений), $U_{\text{эт}}$ - значение напряжения на эталонном резисторе, $U_{\text{нити}}$ - значение напряжения нити, $I_{\text{нити}}$ - сила тока нити, $Q_{\text{нити}}$ - мощность, выделяемая на нити, $R_{\text{нити}}$ - сопротивление нити.

\begin{table}[H]
\caption{Измерения при $T = 22^{\circ}C$}
\label{   }
\begin{tabular}{|c|c|c|c|c|c|c|c|c|}
\hline
№ опыта    & 1        & 2       & 3       & 4       & 5       & 6       & 7       & 8       \\ \hline
$R_{\text{м}}$, Ом    & 1000     & 100     & 90      & 80      & 60      & 40      & 20      & 0       \\ \hline
$U_{\text{эт}}$, В     & 0,039487 & 0,32440 & 0,35265 & 0,38627 & 0,47716 & 0,62360 & 0,89715 & 1,557   \\ \hline
$U_{\text{нити}}$, В   & 0,057518 & 0,47410 & 0,51570 & 0,56530 & 0,70026 & 0,92027 & 1,3428  & 2,467   \\ \hline
$I_{\text{нити}}$, А   & 0,00395  & 0,03244 & 0,03527 & 0,03863 & 0,04772 & 0,06236 & 0,08972 & 0,15570 \\ \hline
$Q_{\text{нити}}$, Вт  & 0,00023  & 0,01538 & 0,01819 & 0,02184 & 0,03341 & 0,05739 & 0,12047 & 0,38411 \\ \hline
$R_{\text{нити}}$, Ом & 14,5663  & 14,6147 & 14,6236 & 14,6348 & 14,6756 & 14,7574 & 14,9674 & 15,8446 \\ \hline
\end{tabular}
\end{table}

\begin{table}[H]
\caption{Измерения при $T = 30^{\circ}C$}
\label{   }
\begin{tabular}{|c|c|c|c|c|c|c|c|}
\hline
№ опыта    & 1        & 2       & 3       & 4       & 5       & 6       & 7       \\ \hline
$R_{\text{м}}$, Ом    & 1000     & 100     & 60      & 30      & 20      & 10      & 0       \\ \hline
$U_{\text{эт}}$, В     & 0,039472 & 0,32349 & 0,47519 & 0,73152 & 0,89024 & 1,1330  & 1,5383  \\ \hline
$U_{\text{нити}}$, В   & 0,058864 & 0,48415 & 0,71410 & 1,1107  & 1,3636  & 1,7647  & 2,4862  \\ \hline
$I_{\text{нити}}$, А   & 0,00395  & 0,03235 & 0,04752 & 0,07315 & 0,08902 & 0,11330 & 0,15383 \\ \hline
$Q_{\text{нити}}$, Вт  & 0,00023  & 0,01566 & 0,03393 & 0,08125 & 0,12139 & 0,19994 & 0,38245 \\ \hline
$R_{\text{нити}}$, Ом & 14,9128  & 14,9665 & 15,0277 & 15,1835 & 15,3172 & 15,5755 & 16,1620 \\ \hline
\end{tabular}
\end{table}

\begin{table}[H]
\caption{Измерения при $T = 40,1^{\circ}C$}
\label{   }
\begin{tabular}{|c|c|c|c|c|c|c|c|}
\hline
№ опыта    & 1        & 2       & 3       & 4       & 5       & 6       & 7       \\ \hline
$R_{\text{м}}$, Ом    & 1000     & 100     & 60      & 30      & 20      & 10      & 0       \\ \hline
$U_{\text{эт}}$, В     & 0,039455 & 0,32233 & 0,47267 & 0,72560 & 0,88160 & 1,11925 & 1,5153  \\ \hline
$U_{\text{нити}}$, В   & 0,060540 & 0,49657 & 0,73111 & 1,1336  & 1,3888  & 1,7915  & 2,5109  \\ \hline
$I_{\text{нити}}$, А   & 0,00395  & 0,03223 & 0,04727 & 0,07256 & 0,08816 & 0,11193 & 0,15153 \\ \hline
$Q_{\text{нити}}$, Вт  & 0,00024  & 0,01601 & 0,03456 & 0,08225 & 0,12244 & 0,20051 & 0,38048 \\ \hline
$R_{\text{нити}}$, Ом & 15,3441  & 15,4056 & 15,4677 & 15,6229 & 15,7532 & 16,0063 & 16,5703 \\ \hline
\end{tabular}
\end{table}

\begin{table}[H]
\caption{Измерения при $T = 50,1^{\circ}C$}
\label{   }
\begin{tabular}{|c|c|c|c|c|c|c|c|}
\hline
№ опыта    & 1        & 2       & 3       & 4       & 5       & 6       & 7       \\ \hline
$R_{\text{м}}$, Ом    & 1000     & 100     & 60      & 30      & 20      & 10      & 0       \\ \hline
$U_{\text{эт}}$, В     & 0,039438 & 0,32118 & 0,47026 & 0,71995 & 0,87329 & 1,1061  & 1,4915  \\ \hline
$U_{\text{нити}}$, В   & 0,062265 & 0,50940 & 0,74864 & 1,1572  & 1,4149  & 1,8193  & 2,5346  \\ \hline
$I_{\text{нити}}$, А   & 0,00394  & 0,03212 & 0,04703 & 0,07200 & 0,08733 & 0,11061 & 0,14915 \\ \hline
$Q_{\text{нити}}$, Вт  & 0,00025  & 0,01636 & 0,03521 & 0,08331 & 0,12356 & 0,20123 & 0,37804 \\ \hline
$R_{\text{нити}}$, Ом & 15,7881  & 15,8603 & 15,9197 & 16,0733 & 16,2019 & 16,4479 & 16,9936 \\ \hline
\end{tabular}
\end{table}

\begin{table}[H]
\caption{Измерения при $T = 60,1^{\circ}C$}
\label{   }
\begin{tabular}{|c|c|c|c|c|c|c|c|}
\hline
№ опыта    & 1        & 2       & 3       & 4       & 5       & 6       & 7       \\ \hline
$R_{\text{м}}$, Ом    & 1000     & 100     & 60      & 30      & 20      & 10      & 0       \\ \hline
$U_{\text{эт}}$, В     & 0,039422 & 0,32005 & 0,46782 & 0,71427 & 0,86504 & 1,0930  & 1,4689  \\ \hline
$U_{\text{нити}}$, В   & 0,063984 & 0,52200 & 0,76587 & 1,1802  & 1,4400  & 1,8458  & 2,5582  \\ \hline
$I_{\text{нити}}$, А   & 0,00394  & 0,03201 & 0,04678 & 0,07143 & 0,08650 & 0,10930 & 0,14689 \\ \hline
$Q_{\text{нити}}$, Вт  & 0,00025  & 0,01671 & 0,03583 & 0,08430 & 0,12457 & 0,20175 & 0,37577 \\ \hline
$R_{\text{нити}}$, Ом & 16,2305  & 16,3100 & 16,3710 & 16,5232 & 16,6466 & 16,8875 & 17,4158 \\ \hline
\end{tabular}
\end{table}

\begin{table}[H]
\caption{Измерения при $T = 70,1^{\circ}C$}
\label{   }
\begin{tabular}{|c|c|c|c|c|c|c|c|}
\hline
№ опыта    & 1        & 2       & 3       & 4       & 5       & 6       & 7       \\ \hline
$R_{\text{м}}$, Ом    & 1000     & 100     & 60      & 30      & 20      & 10      & 0       \\ \hline
$U_{\text{эт}}$, В     & 0,039403 & 0,31888 & 0,46533 & 0,70852 & 0,85666 & 1,0799  & 1,4464  \\ \hline
$U_{\text{нити}}$, В   & 0,065730 & 0,53472 & 0,78320 & 1,2030  & 1,4650  & 1,8720  & 2,5807  \\ \hline
$I_{\text{нити}}$, А   & 0,00394  & 0,03189 & 0,04653 & 0,07085 & 0,08567 & 0,10799 & 0,14464 \\ \hline
$Q_{\text{нити}}$, Вт  & 0,00026  & 0,01705 & 0,03644 & 0,08523 & 0,12550 & 0,20216 & 0,37327 \\ \hline
$R_{\text{нити}}$, Ом & 16,6815  & 16,7687 & 16,8311 & 16,9791 & 17,1013 & 17,3349 & 17,8422 \\ \hline
\end{tabular}
\end{table}

\subsection{Построение нагрузочных кривых}

По полученным данным построены нагрузочные кривые для каждой из исследуемых температур, вычислены свободные коэффициенты графиков, численно равные сопротивлению нити при температуре нити равной температуре термостата. Также вычислены коэффициенты наклона графиков $\frac{dR}{dQ}$ и погрешности полученных значений.

\begin{figure}[H]
\begin{floatrow}
\ffigbox{\caption{}\label{}}
{\includegraphics[scale=0.42]{22°C график}}
\ffigbox{\caption{}\label{}}
{\includegraphics[scale=0.42]{30°С график}}        
\end{floatrow}
\end{figure}	

\begin{figure}[H]
\begin{floatrow}
\ffigbox{\caption{}\label{}}
{\includegraphics[scale=0.42]{40,1°C график}}
\ffigbox{\caption{}\label{}}
{\includegraphics[scale=0.42]{50,1°C график}}        
\end{floatrow}
\end{figure}	

\begin{figure}[H]
\begin{floatrow}
\ffigbox{\caption{}\label{}}
{\includegraphics[scale=0.42]{60,1°C график}}
\ffigbox{\caption{}\label{}}
{\includegraphics[scale=0.42]{70,1°C график}}        
\end{floatrow}
\end{figure}

На основе коэффициентов, полученных при аппроксимации графиков, составлена сводная таблица. 
\begin{table}[H]
\caption{Сводная таблица по всем графикам. $T$ - температура термостата, $R(T)$ - предполагаемое значение сопротивление нити при температуре равной температуре термостата, $\sigma_{R(T)}$ - погрешность полученного сопротивления, $\frac{dR}{dQ}$ - значение коэффициента наклона графика $R(Q)$, $\sigma_{\frac{dR}{dQ}}$ - погрешность коэффициента наклона графика.}
\label{}
\begin{tabular}{|c|c|c|c|c|}
\hline
$T$, $^{\circ}C$ & $R(T)$, Ом & $\sigma_{R(T)}$, Ом & $\frac{dR}{dQ}$, Ом/Вт & $\sigma_{\frac{dR}{dQ}}$, Ом/Вт \\ \hline
22   & 14,5642 & 0,0007 & 3,335 & 0,005 \\ \hline
30   & 14,917  & 0,002  & 3,267 & 0,013 \\ \hline
40,1 & 15,355  & 0,004  & 3,21  & 0,02  \\ \hline
50,1 & 15,805  & 0,005  & 3,16  & 0,03  \\ \hline
60,1 & 16,253  & 0,007  & 3,12  & 0,04  \\ \hline
70,1 & 16,710  & 0,008  & 3,06  & 0,05  \\ \hline
\end{tabular}
\end{table}

\subsection{Исследование зависимости сопротивления нити от температуры}

С помощью таблицы 7 построен график зависимости $R(T)$ и вычислен коэффициент наклона графика $\frac{dR}{dT}$ и температурный коэффициент сопротивления материала нити $\alpha = \frac{1}{R_{273}} \cdot \frac{dR}{dT}$. Графики и полученные коэффициенты представлены ниже.

\begin{center}
\begin{figure}[H]
\floatsetup[wrapfigure]{capposition=bottom}
\includegraphics[scale=0.6]{R(T) график}
\label{1}
\end{figure}
\end{center}

При аппроксимации были получены следующие значения $\frac{dR}{dT} = 0,0446 \pm 0,0002$ Ом/К, $R_{273} = 13,577 \pm 0,009$ Ом, $\alpha = \frac{1}{R_{273}} \cdot \frac{dR}{dT} = (3284 \pm 15) \cdot 10^{-6}$ $1/K$.

\subsection{Вычисление коэффициента теплопроводности для исследуемых температур}

На основе данных, полученных ранее, по формуле (2) был вычислен коэффициент теплопроводности для каждой из исследуемых температур, погрешности полученных результатов вычислены по формуле (3), полученные данные занесены в таблицу.

\begin{table}[H]
\caption{Результаты вычислений коэффициента теплопроводности воздуха и его погрешности при исследуемых температурах. $T$, K - температура термостата, $k$, Вт/(м $\cdot$ К) - коэффициент теплопроводности воздуха, $\sigma_k$, Вт/(м $\cdot$ К) - абсолютная погрешность $k$, $\varepsilon_k, \%$ - относительная погрешность $k$ в процентах}
\label{}
\begin{tabular}{|c|c|c|c|}
\hline
$T$, K   & $k$, Вт/(м $\cdot$ К) & $\sigma_k$, Вт/(м $\cdot$ К)	 & $\varepsilon_k, \%$ \\ \hline
295,15 & 0,0303 & 0,0006        & 1,9         \\ \hline
303,15 & 0,0310 & 0,0006        & 1,9         \\ \hline
313,25 & 0,0315 & 0,0006        & 2,0         \\ \hline
323,25 & 0,0320 & 0,0007        & 2,1         \\ \hline
333,25 & 0,0324 & 0,0007        & 2,3         \\ \hline
343,25 & 0,0330 & 0,0008        & 2,5         \\ \hline
\end{tabular}
\end{table}

\subsection{Определение коэффициента пропорциональности $\beta$: $k \propto T^{\beta}$}
На основе данных таблицы 8 был построен график зависимости $ln(K)(kn(T))$, погрешность температуры в данном случае не учитывалась, потому что она много меньше погрешности коэффициента $k$. Погрешность $\sigma_{ln(k)}$ вычислена по следующей формуле: \begin{large} $\sigma_{ln(k)} = \frac{d(ln(k))}{dk} = \frac{\sigma_k}{k}$ \end{large}

\begin{center}
\begin{figure}[H]
\floatsetup[wrapfigure]{capposition=bottom}
\includegraphics[scale=0.5]{Финальный график}
\label{1}
\end{figure}
\end{center}

При аппроксимации было получено следующее значение коэффициента $\beta = 0,53 \pm 0,03$.

\section{Анализ полученных результатов и выводы}

В данной работе измерена зависимость теплопроводности воздуха от температуры. При обработке экспериментальных данных получены следующие величины: 1) температурный коэффициент сопротивления молибдена $\alpha = (3284 \pm 15) \cdot 10^{-6}$ $1/K$. 2) Коэффициент теплопроводности воздуха $k \in [0,0300; 0,0330] \pm 0,0008$ Вт/(м $\cdot$ K) при $T \in [20^{\circ}C; 70^{\circ}C]$. 3) Коэффициент пропорциональности $\beta$ между $k$ и $T$ ($\beta$: $k \propto T^{\beta}$): $\beta = 0,53 \pm 0,03$.

Таблицное значение коэффициента $\alpha = 0,004579$, полученное экспериментальным путём значение совпадает с ним лишь по порядку величины. Такое расхождение может быть связано как с наличием примесей в материале нити, так и с ошибкой в методике измерений. 

Табличное значение коэффициента теплопроводности воздуха в данном интервале температур $k_{\text{табл}} = (2,6 \sim 3) \cdot 10^{-2}$Вт/(м $\cdot$ K). Полученные значения коэффициента теплопроводности примерно совпадают с табличными, погрешность составляет около $15 \%$. Поэтому данный метод позволяет вычислить коэффициент теплопроводности воздуха с точностью $15 \%$. 

Теоретическое значение коэффициента $\beta_{\text{теор}} = 0,5$, оно лежит в пределах погрешности $\beta_{\text{изм}}$. Таким образом, получившееся значение $\beta$ согласуется со значением, полученным теоретически, если считать молекулы твёрдыми шариками.

Из всего этого можно сделать вывод, что данный метод применим к исследованию теплопроводности воздуха. Точность измерения коэффициента теплопроводности воздуха составляет $15 \%$. Вычисленное на основе экспериментальных данных значение коэффициента пропорциональности между $k$ и $T$ согласуется со значением, предсказанным теорией, в которой молекулы считаются твёрдыми шариками.
\end{document}